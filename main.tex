% ! TEX program = xelatex
\documentclass{cdutproposal}
% 盲审模式
\usepackage{mathtools}
\usepackage[most]{tcolorbox}
\usepackage{geometry}
\usepackage{enumitem}

% Set margins to include tcolorbox frame
\newlength{\framewidth}
\setlength{\framewidth}{1.5pt}  % Frame thickness
\newlength{\framepadding}
\setlength{\framepadding}{5pt}  % Padding

\geometry{
  top=2.0cm,
  bottom=\dimexpr 2.2cm+2\framewidth+2\framepadding,
  left=2.4cm,
  right=\dimexpr 2.4cm+2\framewidth+2\framepadding
}

\addbibresource{ref.bib}

\begin{document}

\thisfancypage{}{}
\thispagestyle{empty}                                       % 清空页面格式
\begin{center}
\fontsize{14pt}{18pt}\selectfont {成都理工大学} \\
学生学士学位论文开题报告
\end{center}

\begin{tcolorbox}[
  blanker,  % Invisible box (only frame is drawn)
  width=\textwidth,
  height=0.8\textheight,
  left=\framepadding,
  right=\framepadding,
%   top=\framepadding,
  bottom=\framepadding,
  borderline west={1.5pt}{0pt}{black},  % Left frame only
  borderline east={1.5pt}{0pt}{black},
  borderline north={1.5pt}{0pt}{black},
  borderline south={1.5pt}{0pt}{black},
  ]
	\zihao{5}%
	\hspace{-0.18cm}
	\begin{tblr}{
	width = 1.02\linewidth,  % 表格与文本同宽
    colspec = {X[3.7,l]|X[3.7,l]|X[2.6,l]}, % 调整比例
    hlines, % 水平线
    vlines, % 垂直线
	}
	\SetCell[c=2]{l} 题目名称:基于xxxxxxxxxxxxxxxxxx研究与实现 && 题目类型: 应用研究 \\ 
	\SetCell[c=2]{l} (Study on XXXXX) && 题目来源: 教师拟定 \\ 
	学生姓名: 张三 & 学生学号: 202305050000 & 专业名称: 地球物理学 \\ 
	导师姓名: 李四 & 专业职称: 教授 & 指导人数: 3 \\ 
\end{tblr}


\section{主要研究内容及预期成果:}
\textbf{主要研究内容}: 
\begin{enumerate}
\item 系统查阅国内外关于...
\item 学习...
\end{enumerate}

\textbf{预期成果}:
\begin{enumerate}
\item 系统掌握...;
\item 实现基于...。
\end{enumerate}

\section{研究思路:}
\textbf{研究方法}:

\textbf{技术路线}:

\textbf{可行性论证}:
\hruleinbox
\section{现有研究基础:}
\textbf{毕业实习}:
在成都理工大学校内,利用学校图书馆等线上资源进行文献收集调研,后续通过... 。

\textbf{资料收集情况}:
收集了前人关于... 。

\textbf{空间设备仪器条件}:
笔记本电脑(LAPTOP-PF2ZE6H6)、office软件、Matlab、Python等高级程序语言。


\end{tcolorbox}

\clearpage

\section{主要参考文献目录及文献综述:}
\subsection{立题意义}

\subsection{国内外的研究现状}

\subsection{参考文献}
\printbibliography[title=主要参考文献]
\clearpage

\section{工作计划}
\smallskip
\begin{nomargin}[h]
	\zihao{5}%
\hspace{-0.15cm}
	\begin{tblr}{
	width = 1.005\linewidth,  % 表格与文本同宽
    colspec = {X[3,l]|X[2,l]|X[1.5,l]|X[1.5,l]}, % 调整比例
    hlines, % 水平线
    vlines, % 垂直线
}
    \SetCell[c=1]{c}起止日期            & \SetCell[c=1]{c}主要任务               & \SetCell[c=1]{c}工作地点         & \SetCell[c=1]{c}联系方式       \\
    2025年01月01日—01月31日 & 选题                  & 成都理工大学     & 151XXXX1750   \\
    2025年02月01日—02月29日 & 资料收集、预研        & 成都理工大学     & 151XXXX1750   \\
    2025年03月01日—03月31日 & 开题报告              & 成都理工大学     & 151XXXX1750   \\
    2025年04月01日—05月10日 & 开展研究、撰写论文    & 成都理工大学     & 151XXXX1750   \\
    2025年05月11日—05月31日 & 完成修改、定稿        & 成都理工大学     & 151XXXX1750   \\
    2025年06月01日—06月09日 & 答辩                  & 成都理工大学     & 151XXXX1750   \\
\end{tblr}
\end{nomargin}

\section{指导教师或指导小组评价(题目、工作要点、方法、进度及准备情况):}
\vspace{5cm}
\quad\quad \quad\quad\quad \quad\quad\quad 指导教师(签名):   \quad\quad\quad \quad\quad\quad      \quad\quad\quad                    年\quad\quad 月\quad\quad 日
\hruleinbox
\section{对学生开题报告的评审意见(是否同意进入毕业论文或毕业设计撰写阶段):}
\vspace{0.5cm}

\quad\quad \quad\quad	同意指导教师意见,同意开题。

\vspace{2.5cm}

\quad\quad \quad\quad\quad \quad\quad\quad 教学系主任(签字):    \quad\quad\quad \quad\quad\quad      \quad\quad                      年  \quad\quad 月 \quad\quad  日

\end{document}

% vim: ts=2 sw=2
